\chapter{Conclusion}
\label{chap:conclusion}

In this work, an extensible and general-purpose crowdsourcing framework is defined and a platform is developed to solve sophisticated problems. The framework allows users to define real-world problems using a component-based model and implement a solution by creating a crowdsourcing application with ease. Platform manages the coordination between computer and human components effectively and produces the results that user is asking for. Based on the concepts of stream processing, $Crowdy$ provides an efficient way to describe problems by employing various components and managing the flow of information and dependencies between them.

Two case studies and multiple experiments show how the component-based framework can handle complex and sophisticated problems such as translation. In these studies, it is proved that the platform can 


There are a number of possible improvement aspects that can be considered as future work. Considering the limitations 

Further, the platform can be possibly integrated into other task markets and crowdsourcing services. Although this work utilizes Amazon?s MTurk as a resource for human computation, there are other available services such as CrowdFlower or oDesk that can be applied. If it is possible to apply the platform to the service where the details of each individual worker better known, that might lead better and greater opportunities to manage resource allocation and task assignment.

