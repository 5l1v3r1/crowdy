\chapter{Introduction} 

%% Background

The growing software systems are characterized by asynchrony, 
heterogeneity and inherent loose coupling promoting system of systems 
as a natural design abstraction. The new system concept goes beyond 
the size of current definition by several measures such as 
number of people the system employed for different purposes; 
number of connections and interdependencies among components; 
number of hardware elements; 
amount of data stored, accessed, manipulated, and refined 
and number of lines of code~\cite{ULSReport}. These requirements lean 
towards a decentralized and dynamic structure that is formed by various 
systems interacting in complex ways.

Therefore, software system becomes an ecosystem in which components supported by 
a common platform operate through exchange of information, resources and 
artifacts and contribute to the overall service that system tries to provide~\cite{Ecosystem}. 
In fact, the components that are fundamental to system 
functionalities are not only software components, but there are now components 
whose functionality is operated by human beings. People become not only users, 
but also an integral part of the system providing 
content and computation, and the overall behavior~\cite{ULSReport}.

Human involvement not only makes the system gain a social characteristic 
in addition to it's technical features, it also gives ability to solve numerous 
problems, including the ones requiring human intelligence. In that sense, the scale 
and variety of components involved within the system increases significantly, and 
homogeneity of components cases respectively. The difference between 
the roles concerning system components and 
humans (user, developer) becomes less distinct. Humans take an essential 
part of the system in collaboration with software components.

The scale of collaboration of creative and cognitive people with 
number-crunching computer systems has expanded from small or 
medium size to internet-scale~\cite{Dorn2012b} leading to new era of 
computation. Although this collaboration has appeared under many names 
such as human computation, collective intelligence, social computing, 
global brain etc, crowdsourcing is the main term that is being used to refer to 
human and computer collaboration.

Crowdsourcing as the new and powerful mechanism of computation has become 
compelling to accomplish work online~\cite{Kittur2011}. Over the past decade, numerous 
crowdsourcing systems have appeared on the Web (Threadless, iStockphoto, 
InnoCentive etc). Such systems enable excessive collaboration of people have 
provided solutions to the problems and tasks that are trivial for humans, which 
cannot be easily completed by computers or computerized. Hundreds of thousands 
of people have worked on various tasks including deciphering scanned text (recaptcha.net), 
discovering new galaxies (galaxyzoo.org), seeking missing people (helpfindjim.com), 
solving research problems (Innocentive), designing t-shirts (Threadless). Even 
Wikipedia and Linux can be viewed as crowdsourcing systems from a point of view that
conceives crowdsourcing as explicit collaboration of users to 
build a long-lasting and beneficial artefact~\cite{Doan2011}.

%% Problem

\section{Problem Statement}
\label{sec:problem}

Although current crowdsourcing systems allow a variety of tasks 
to be completed by people, the tasks requested for completion are typically simple. 
Tasks, often described as micro-tasks, have the two following fundamental characteristics:

\textbf{Difficulty.} Tasks are narrowly focused, low-complex and require 
little expertise and cognitive effort to complete (taking a couple of seconds to 
a few minutes).

\textbf{Dependency.} Tasks assigned to humans are independent of each other. 
The current state of one job has no effect on the other. The result of one job cannot 
be input to the other to create some information flow.

In that sense, simplicity makes the division and distribution of tasks among 
individuals easy~\cite{Zhang2011}, and independency enables parallelizing 
and bulk-processing tasks. However, solving more complex and sophisticated 
problems requires effective and efficient coordination of computation sources 
(human or software) rather than creating and listing a series of micro-tasks 
to-be-completed.

Recently detailed analysis on current mechanisms based on foundations of 
crowdsourcing reveal the necessity of a more sophisticated problem-solving 
paradigm~\cite{Kittur2013}. Researchers explicitly state the need for a new generic platform 
with the ability to tackle advanced problems. Kittur et al.~\cite{Kittur2013} 
suggest researchers to form new concepts of crowd work beyond 
the simple, independent and deskilled tasks. Based on the fact that complex 
work cannot be accomplished via existing simple and parallel approaches,
the authors state the requirement for a platform to design multi-stage workflows 
to complete complex tasks, which can be decomposed into smaller subtasks, 
by appropriate groups of workers selected through a set of constraints.

In another piece of work, Bernstein et al.~\cite{Bernstein2012} regard all the 
people and computers as constituting a $global$ $brain$. Authors indicate the 
need for new powerful programming metaphors that can more accurately 
demonstrate the way people and computer work in collaboration. These 
metaphors are expected to solve dependent sections of more complex problems by 
decompositions and management of interdependencies. Further, the specification 
of task sequence and information flow are expected to enable deliberate 
collaboration over solutions.

However, recent research only partially addresses these challenges by 
providing programming frameworks and models
~\cite{Kittur2011, Ahmad2011, Kokciyan2012, Little2009, Minder2011, Barowy2012, Kulkarni2012, Kittur2012} 
for massively parallel human computation, limited-scope and ability user interfaces
~\cite{Marcus2011, Bernstein2010, Marcus2011b, Rzeszotarski2012}, 
concepts for planning~\cite{Zhang2012}, 
analysis of collaboration~\cite{Dorn2012}. 
These studies fail to tackle challenges of crowdsourcing due to following reasons:
\begin{itemize}
	\item having rigid structure and requirements due to the (programming) concepts and 
	libraries that they are based on
	\item being only applicable to a small and bounded problem-set
	%\item focusing on a specific aspect of crowdsourcing
	%\item being developed in an ad hoc manner
	\item requiring a significant amount of work in order to implement and integrate 
	a crowdsourcing solution to solve a problem.
\end{itemize}

%These works and related issues are further 
%explained and examined thoroughly in the following section.

Further, human workers are often regarded as homogeneous and 
interchangeable due to the issues of scalability and availability in existing 
mechanisms~\cite{Ahmad2011}. However, people in a crowd have different 
skills, and can perform different roles based on their interests and expertise~\cite{Zhang2011}. 
Current services are created without considering the availability and preferences of 
people, constraints and relationships, and the support of dynamic 
collaborations~\cite{Schall2010}. Thus, human involvement in current mechanisms should 
be rethought due to limited support for collaboration and 
ignorance of collaboration patterns in problem-solving~\cite{Dorn2012a} 
over general-purpose infrastructures that can more accurately reflect 
the collaboration of people and computers~\cite{Bernstein2012, Minder2012}.

%This additionally restricts 
%the scope and complexity of tasks that can be addressed by crowdsourcing systems, and 
%disregards one of the main characteristics of crowdsourcing, which is the diversity 
%of computation sources.

%Again there is a fundamental requirement for general-purpose 
%infrastructures that can more accurately reflect the collaboration 
%of people and computers~\cite{Minder2012, Bernstein2012}.
 
Nevertheless, the development of more generic crowdsourcing platforms 
along with new applications and structures are expected 
by the research community~\cite{Doan2011}.

%% Purpose

\section{Purpose}

This study aims to solve the previously mentioned issues associated with 
crowdsourcing platforms and tackle challenges in crowdsourcing. In order to 
achieve this, the existing frameworks and platforms are identified, analyzed 
and discussed thoroughly. Based on the data and feedback extracted out of 
the related studies, a new general-purpose and extensible framework is 
proposed. The framework is implemented into a tool and evaluated through 
real-world case studies.

The proposed platform, called $Crowdy$, is developed to support software 
ecosystems via stream-based human computation. $Crowdy$ can be used 
to design and develop crowdsourcing applications for effective and efficient 
collaboration of human and software components. The platform
\begin{itemize}
	\item enables users to perform computations to solve complex problems
	\item has no rigid structure or requirements
	\item is not limited to a specific problem-set or aspect of crowdsourcing
	%\item does not require significant amount of work to implement and solve the problems.
\end{itemize}

The platform consists of an application editor, a runtime 
environment and computation resources. Users design applications by simply 
creating and connecting operators together. These applications are submitted to runtime 
environment. The runtime environment executes applications by creating processes. 
A process is performed via corresponding computation resources, which can be either 
people or software. In the case of human computation existing crowdsourcing services 
are used. Otherwise, computers are utilized.

This platform is concentrated on providing mechanisms that can be used to decompose 
the implementation of an application into a set of components as a close representation 
of the real-world problem. The main characteristic of this work is to show how 
sophisticated problems can be accomplished cleanly and easily relying on 
component-based model.

%% Thesis Plan

\section{Thesis Plan}

The remainder of the work continues as follows. In Chapter \ref{chap:background}, 
background information on crowdsourcing is given. Chapter \ref{chap:architecture} 
introduces the architecture of proposed platform. Chapter \ref{chap:platform} provides 
the implementation details of the proposed platform and concepts used in this work. 
In Chapter \ref{sec:tool}, the tool developed over the platform is demonstrated. 
Various motivating scenarios are described and developed using the proposed 
platform in Chapter \ref{chap:examples}. Related studies are examined in Chapter 
\ref{chap:relatedwork}. In Chapter \ref{chap:discussion}, a brief discussion of future work 
is given. A final chapter concludes this study.
