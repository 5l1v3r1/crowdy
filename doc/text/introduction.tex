\chapter{Introduction} 

%% Background

The asynchrony, heterogeneity and inherent loose coupling that characterize applications 
promote system of systems as a natural design abstraction for growing class of software 
systems. This new concept goes beyond the size of current system definition by several 
measures such as 
number of people the system employing for different purposes; 
number of connections and interdependencies among components; 
number of hardware elements; 
amount of data stored, accessed, manipulated, and refined 
and number of lines of code. Such systems are called 
Ultra-Large-Scale (ULS) systems and perceived as socio-technical 
ecosystems~\cite{ULSReport}.

The socio-technical aspect of an ULS system is a result of decentralized and dynamic 
structure formed by people and software components interacting in complex ways. 
People become not only users, but also an integral part of the system providing 
content and computation, and the overall behavior~\cite{ULSReport}. 
The difference between the roles concerning system components and 
humans (user, developer) becomes less distinct. The homogeneity 
of components ceases together with the increasing scale and variety of people 
and software components involved within the system. Thus, system gains a social 
and technical characteristic with the ability to solve numerous problems, 
even the ones requiring human intelligence.

Social aspect of systems has expanded the scale of collaboration from small or 
medium sized to internet-scale~\cite{Dorn2012b} leading to new era of computation. 
Collaboration of creative and cognitive people with number-crunching 
computer systems have appeared under many names such as crowdsourcing, 
human computation, collective intelligence, social computing, global brain etc, 
for which you can find detailed studies on classification of systems and ideas 
in~\cite{Quinn2009, Quinn2011} collected under distributed human computation term.

The term $crowdsourcing$, which is the main consideration in this body of work, 
is first coined by Jeff Howe in the June 2006 issue of Wired magazine~\cite{Howe2006b} 
as an alternative to the traditional, in-house approaches focusing on assigning tasks 
to employees in the company for solving problems. Crowdsourcing describes a new 
mainly web-based business model that exploits collaboration of individuals in a 
distributed network through an open call for proposals. The term is described by Howe as:

\begin{quotation}
Simply defined, crowdsourcing represents the act of a company or institution taking 
a function once performed by employees and outsourcing it to an undefined 
(and generally large) network of people in the form of an open call. This can take 
the form of peer-production (when the job is performed collaboratively), but is also 
often undertaken by sole individuals. The crucial prerequisite is the use of the open 
call format and the large network of potential laborers.~\cite{Howe2006a}
\end{quotation}

Crowdsourcing as the new and powerful mechanism of computation has become 
powerful to accomplish work online~\cite{Kittur2011}. Over the past decade, numerous 
crowdsourcing systems have appeared on the Web (Threadless, iStockphoto, 
InnoCentive etc). Such systems enabling excessive collaboration of people have 
provided the solutions to the problems and tasks that are trivial for humans, but 
cannot be easily completed by computers or computerized. Hundreds of thousands 
of people have worked on various tasks including deciphering scanned text (recaptcha.net), 
discovering new galaxies (galaxyzoo.org), seeking missing people (helpfindjim.com), 
solving research problems (Innocentive), designing t-shirts (Threadless). Even 
Wikipedia and Linux can be viewed as crowdsourcing systems from a point of view that
conceives crowdsourcing as explicit collaboration of users to 
build a long-lasting and beneficial artefact~\cite{Doan2011}.

Indeed, one of the most interesting developments in terms of crowdsourcing is 
Amazon's Mechanical Turk (MTurk), which is a general purpose crowdsourcing platform 
recruiting large numbers of people to complete diverse jobs. Assignments on MTurk 
range from labeling images with keywords, transcribing an audio snippet, finding some 
piece of information on the Web. Requesters submit jobs, which are called Human 
Intelligent Tasks or HITs in MTurk parlance, as HTML forms. Respectively workers, 
who are the crowd of users, (called Turkers on MTurk) perform or complete these 
jobs by inputting their answers receiving a small payment in return. This actual 
platform and other example systems listed above present the potential to accomplish 
work in different areas within less time and money required by traditional 
methods~\cite{Minder2012, Marcus2011}.

%% Problem

\section{Problem}
\label{sec:problem}

Although current crowdsourcing systems led by MTurk allow a variety of tasks 
to be completed by people, the tasks requested for completion are typically simple. 
Tasks, often described as micro-tasks, have the following two fundamental characteristics:

\textbf{Complexity.} Tasks are narrowly focused, low-complex and require 
little expertise and cognitive effort to complete (taking a couple of seconds to 
a few minutes).

\textbf{Dependency.} Tasks assigned to humans are independent of each other. 
The current state of one job has no effect on the other. The result of one job cannot 
be input to the other to create some information flow.

In that sense, simplicity makes the division and distribution of tasks among 
individuals easy~\cite{Zhang2011}, and independency enables parallelizing 
and bulk-processing tasks. However, solving more complex and sophisticated 
problems requires effective and efficient coordination of computation sources 
(human or software) rather than creating and listing a series of micro-tasks 
to-be-completed.

Recently detailed analyses on current mechanisms based on foundations of 
crowdsourcing reveal the necessity of a more sophisticated problem-solving 
paradigm. Researchers explicitly state the need for a new generic platform 
with ability to tackle advanced problems. Kittur et al.~\cite{Kittur2013} 
suggest researchers to form new concepts of crowd work beyond 
the simple, independent and deskilled tasks. Based on the fact that complex 
work cannot be accomplished via existing simple and parallel approaches,
the authors state requirement for a platform to design multi-stage workflows 
to complete complex tasks, which can be decomposed into smaller subtasks, 
by appropriate groups of workers selected through a set of constraints.

In another piece of work, Bernstein et al.~\cite{Bernstein2012} regard all the 
people and computers as constituting a $global$ $brain$, and they indicate the 
need for powerful new programming metaphors that can more accurately 
demonstrate the way people and computer work in collaboration. These 
metaphors are expected to solve dependent sections of more complex problems by 
decompositions and management of interdependencies. Further, the specification 
of task sequence and information flow are expected with deliberate collaboration 
over solutions.

However, recent research only partially addresses these challenges by 
providing programming frameworks and models
~\cite{Kittur2011, Ahmad2011, Kokciyan2012, Little2009, Minder2011, Barowy2012, Kulkarni2012, Kittur2012} 
for massive parallel human computation, limited-scope and ability user interfaces
~\cite{Marcus2011, Bernstein2010, Marcus2011b, Rzeszotarski2012}, 
concepts for planning~\cite{Zhang2012}, 
analysis of collaboration~\cite{Dorn2012}. 
These works fail to tackle challenges of crowdsourcing due to various reasons: 
having rigid structure and requirements, being only applicable to a small and 
bounded problem-set, focusing on a specific aspect of crowdsourcing, 
being developed in ad hoc manner, requiring a significant amount of work in 
order to implement and integrate.

%These works and related issues are further 
%explained and examined thoroughly in the following section.

Further, human workers are often regarded as homogeneous and 
interchangeable due to the issues of scalability and availability in existing 
mechanisms~\cite{Ahmad2011}. However, people in a crowd have different 
skills, and can perform different roles based on their interests and expertise~\cite{Zhang2011}. 
Current services are created without considering the availability and preferences of 
people, constraints and relationships, and the support of dynamic 
collaborations~\cite{Schall2010}. Thus, human involvement in current mechanisms should 
be rethought due to limited support for collaboration and 
ignorance of collaboration patterns in problem-solving~\cite{Dorn2012a} 
over general-purpose infrastructures that can more accurately reflect 
the collaboration of people and computers~\cite{Minder2012, Bernstein2012}.

%This additionally restricts 
%the scope and complexity of tasks that can be addressed by crowdsourcing systems, and 
%disregards one of the main characteristics of crowdsourcing, which is the diversity 
%of computation sources.

%Again there is a fundamental requirement for general-purpose 
%infrastructures that can more accurately reflect the collaboration 
%of people and computers~\cite{Minder2012, Bernstein2012}.
 
Nevertheless, the development of more generic crowdsourcing platforms 
along with new applications and structures are expected 
by the research community~\cite{Doan2011}.

%% Purpose

\section{Purpose}

To address the issues, a new extensible general-purpose platform for crowdsourcing 
is proposed in this work. The new platform, $Crowdy$, is a computing stack that 
consists of a model-based framework, and on top of that there is the runtime environment 
supporting design and development of effective and efficient collaboration of
human and software components. This platform is concentrated on providing 
mechanisms that can be used to decompose the implementation of an application 
into a set of components as a close representation of the real-world problem. 
The main characteristic of this work is to show how sophisticated problems can be 
accomplished cleanly and easily relying on component-based model.

%% Thesis Plan

\section{Thesis Plan}

The remainder of the work continues as follows. In Chapter \ref{chap:background}, background information related to this study is given and previous works are examined. Chapter 3 provides the details of the proposed platform. Motivating examples are described and developed using the proposed platform in Chapter 4. In Chapter 5, a brief discussion of future work is given. A final chapter concludes this study.
