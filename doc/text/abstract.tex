\thispagestyle{plain}
\begin{abstract}
The scale of collaboration between people and computers has expanded leading to new era of computation called crowdsourcing. A variety of problems can be solved with this new approach by employing people to complete tasks that cannot be computerized. However, the existing approaches are focused on simplicity and independency of tasks that fall short to solve complex and sophisticated problems that crowdsourcing is challenged. We present Crowd, a general purpose and extensible crowdsourcing platform that lets users to perform computations to solve complex problems using both computers and human workers. Platform is developed over the fundamentals of stream-processing paradigm in which operators execute on the continous stream of data elements. The proposed architecture provides a standart toolkit of operators for computation and configuration support to control and coordinate the solution. There is no rigid structure or requirement that could limit the problem-set, which can be solved with the stream-based approach. This approach is tested and verified over different scenarios. Results show that sophisticated problems can be easily solved without significant amount of work for implementation. Also possible improvements are discussed and identified that is a promising future work for the existing work.
\end{abstract}

\begin{ozet}
 T\"urk\c ce \"ozet buraya gelecek. 
\end{ozet}