\thispagestyle{plain}
\begin{abstract}
The scale of collaboration between people and computers has expanded 
leading to new era of computation called crowdsourcing. A variety of problems 
can be solved with this new approach by employing people to complete tasks 
that cannot be computerized. However, the existing approaches are focused 
on simplicity and independency of tasks that fall short to solve complex and 
sophisticated problems. We present Crowdy, a general-purpose and extensible 
crowdsourcing platform that lets users perform computations to solve complex 
problems using both computers and human workers. The platform is developed 
based on the stream-processing paradigm in which operators execute on the 
continuos stream of data elements. The proposed architecture provides a standard 
toolkit of operators for computation and configuration support to control and 
coordinate resources. There is no rigid structure or requirement that could limit 
the problem-set, which can be solved with the stream-based approach. 
The stream-based human-computation approach is implemented and verified 
over different scenarios. Results show that sophisticated problems can be easily 
solved without significant amount of work for implementation. Also possible 
improvements are discussed and identified that is a promising future work 
for the existing work.
\end{abstract}

\begin{ozet}
\.{I}nsanlar ve yaz{\i}l{\i}m bile\c{s}enleri aras{\i}ndaki i\c{s}birli\u{g}i geli\c{s}erek 
kitle kayna\u{g}{\i}n ortaya \c{c}{\i}kmas{\i}n{\i} sa\u{g}lam{\i}\c{s}t{\i}r. 
Kitle kaynak kullan{\i}larak yaz{\i}l{\i}m taraf{\i}ndan \c{c}\"{o}z\"{u}lemeyen veya \c{c}\"{o}z\"{u}lmesi zor bir\c{c}ok sorun insanlar arac{\i}l{\i}\u{g}{\i} 
ile \c{c}\"{o}z\"{u}lm\"{u}\c{s} ve bir sonuca ula\c{s}{\i}lm{\i}\c{s}t{\i}r. Fakat g\"{u}n\"{u}m\"{u}zdeki kitle kaynak odakl{\i} yakla\c{s}{\i}mlar ve \c{c}\"{o}z\"{u}m s\"{u}re\c{c}leri 
yap{\i}lan i\c{s}lerin basitli\u{g}ine ve birbirlerinden ba\u{g}{\i}ms{\i}z olmalar{\i}na a\u{g}{\i}rl{\i}k vermektedir. Bu sebepten dolay{\i} zor 
ve \c{c}ok y\"{o}nl\"{u} sorunlar{\i}n \c{c}\"{o}z\"{u}m\"{u} mevcut yakla\c{s}{\i}mlarla m\"{u}mk\"{u}n de\u{g}ildir. Bu \c{c}al{\i}\c{s}mada kullan{\i}c{\i}lar{\i}n sorun \c{c}\"{o}z\"{u}m\"{u} 
konusunda hem insanlar{\i} hem de yaz{\i}l{\i}m bile\c{s}enlerini kullanabilecekleri, genel ama\c{c}l{\i} ve geli\c{s}tirilebilir 
bir yakla\c{s}{\i}m ve bu yakla\c{s}{\i}m{\i}n uyguland{\i}\u{g}{\i} bir altyap{\i} sunulmaktad{\i}r. Yakla\c{s}{\i}m k\"{u}\c{c}\"{u}k i\c{s}lemcilerin s\"{u}rekli olarak 
akan veriler \"{u}zerinde \c{c}al{\i}\c{s}mas{\i} mant{\i}\u{g}{\i}na dayanmaktad{\i}r. Sunulan altyap{\i} b\"{u}nyesinden bar{\i}nd{\i}rd{\i}\u{g}{\i} temel i\c{s}lemciler 
sayesinde i\c{s}lem kaynaklar{\i} aras{\i}ndaki kontrol ve e\c{s}g\"{u}d\"{u}m\"{u} kolayl{\i}kla sa\u{g}lamaya elveri\c{s}li \c{s}ekilde tasarlanm{\i}\c{s}t{\i}r. 
Sunulan yakla\c{s}{\i}m k{\i}s{\i}tl{\i} bir sorun listesini hedeflememektedir ve kullan{\i}c{\i}lar a\c{c}{\i}s{\i}ndan herhangi bir k{\i}s{\i}tlama 
getirmemektedir. \c{C}e\c{s}itli \"{o}rnekler yap{\i}lan incelemeler sunulan yakla\c{s}{\i}m{\i}n sorunlar{\i} kayda de\u{g}er bir i\c{s} y\"{u}k\"{u} 
getirmeden \c{c}\"{o}z\"{u}lebilece\u{g}ini g\"{o}stermi\c{s}tir. Ayr{\i}ca, \c{c}e\c{s}itli iyile\c{s}tirme \"{o}nerileri de tart{\i}\c{s}{\i}lm{\i}\c{s} ve baz{\i}lar{\i} 
gelecek \c{c}al{\i}\c{s}malara eklenmek \"{u}zere belirlenmi\c{s}tir.
\end{ozet}