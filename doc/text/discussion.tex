\chapter{Discussion}
\label{chap:discussion}

$Crowdy$ fundamentally depends on the idea that real-world problems can be demonstrated using component-based model. This dependency requires that a complex problem can be broken into pieces of smaller tasks and these tasks can be coordinated to solve the overall problem. However, there may be cases when this assumption can be violated. It is possible to have some work that may not be easily divided into smaller unit of work. For example, asking people's ideas on several topics cannot be easily divided into several tasks. However, this case can be still tackled by $Crowdy$ application by having a source operator and a sink operator in which the data tuples containing various ideas generated by human workers via human source operator are written to a file by sink file operator or emailed to the requester by sink email operator. It is still possible that the decomposition of problem into various tasks and the implementation of those tasks over operators along with necessary quality control steps could introduce some overhead and cost. Nevertheless, the premise of $Crowdy$ is still valid by enabling human-computer collaboration over a component-based model without going through manual processes. 

Secondly, the existing platform does not fully support iteration. Although specific set of operators (human operator, enrich operator) have iteration support using that a task can be copied and processed for a number of times without a need to copy the actual operator, an output of an operator cannot input another operator that is up in the flow. In other words, loops are not allowed. This limitation is due to the fact that data tuples flow through the application from source to sink operators. The full iteration support is left open for the future discussions and improvements on the platform.

Another limitation is the scope of human collaboration. $Crowdy$ supports collaboration of tasks that are completed individually by human workers. The collaborative task completion for which more than one person work on a task together in real-time is not supported. Real-time collaboration is beyond the scope of this work. However, a task or a piece of work can be completed by more than one person. Output of a human task can be taken and input to another human task. In this way, people collaborate with each other to complete a piece of work, but this is not happening in real-time. Even though interdependency of tasks can be easily handled by $Crowdy$, human workers should be provided with enough context and information on the problem that they are collaborating to achieve the best results.

Quality control is the most important challenge for crowdsourcing systems. Since low quality submissions for human tasks is common, most of the quality control efforts focus on human-related computation. Currently quality control can be achieved by adding extra human operators in which we can employ other people to check whether a completed human task has a good quality or not. However, the low quality submissions are still possible for the human tasks to evaluate quality of work done by others. Possible improvement aspects on this issue is further discussed in Chapter~\ref{chap:conclusion}.